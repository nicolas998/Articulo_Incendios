%%%%%%%%%%%%%%%%%%%%%%%%%%%%%%%%%%%%%%%%%%%%%%%%%%%%%%%%%%%%%%%%%%%%%%%%%%%%%%%%
\section{INTRODUCTION}
Forest fires at the Aburra Valley watershed (VA) have caused several forest losses (CITAS), health issues for the residents (CITAS) and increases in the $CO_2$ emissions \citep{Andreae2001}. At the VA, human activities along with weather oscillations affect the occurrence of forest fires, and the efforts to reduce them have shown limited success due to access and public order issues (hay CITAS?). The mentioned characteristics limit event attention and increase the need for prevention strategies.  This prevention goes from educational strategies (CITAS) to technological implementations such as cameras, human monitoring, and models (CITES).\\

With the use of models, it is possible to produce predictions that eventually help the prevention strategies CITAS.  However, according to several studies (CITAS) forest fires vulnerability is explained by the vegetation and its moisture, the weather, the hydrology, and human activities. Each one of the described factors includes its variability and uncertainties that decreases its predictability from a physical point of view CITAS. Due to this, several authors have tried to develop empirical forest fires prediction models (CITAS). HABLAR DE LAGUNOS DE LOS MODELOS QUE HAY.\\

Along with the models, there are efforts in order to observe, monitor, and measure forest fires.  At a global scale, \citep{Fearnside2018} used Landsat and OLI satellite images to monitor forest fires on the Amazonas basin for 33 years. In a more local case, \citep{Valero2017} use unmanned vehicles to track on live fires and eventually improve forest fires accuracy. There are also efforts to analyze detailed information about forest fires at pine forest \citep{ElHoussami2017}, grasslands \citet{Roger2015}, shrublands \citet{Santoni2006}, and forested environments \citet{Wotton2012}. Besides, there is also fire live monitoring by using cameras \citep{Bao2015}. Some of these monitoring efforts involve multiple methodologies CITAS and are usually used along with models for decision making CITAS.\\

In the VA region, forest fires are likely to happen during the dry season of December, January, and February. Due to the land cover and topographical characteristics, forest fires last between several hours to 2 days.  Moreover, their covered area is limited to 30 $ha$.  Considering these characteristics, we develop a strategy that involves monitoring, modeling, and warning. The monitoring consists of 12 forest fires detection cameras along with three high-resolution thermal cameras distributed along the valley.  Also, we develop a Bayesian dynamic model that predicts forest fire vulnerability distributed and at an hourly scale. For the setup of the model, we use static and dynamic information.   Through the cameras and the model, a continuous monitoring strategy is established, which is transmitted to the authorities in real time.

PARRAFO CON LA ESTRUCTURA DEL ARTICULO.



